\chapter{Исследовательский раздел}

\section{Технические характеристики}

Технические характеристики устройства, на котором выполнялось тестирование:
\begin{enumerate}
	\item операционная система~---~macOS 15.1.1 (24B91)~\cite{macos}; 
	\item объем оперативной памяти~---~18 Гбайт;
	\item процессор~---~Apple M3 Pro, 11 ядер~\cite{macos}.
\end{enumerate}

Во время тестирования ноутбук был подключен к сети электропитания и нагружен только встроенными приложениями и системой тестирования.

\section{Описание исследования}

В качестве стороннего веб-сервера для сравнения производительности был выбран \texttt{nginx}, так как является самым популярным и востребованным~\cite{nginx}.
В качестве инструмента для генерации нагрузки и замеров производительности будет использоваться утилита \texttt{ab} (Apache Benchmark)~\cite{ab}.
Целью исследования является сравнение реализованного веб-сервера с известными аналогами по времени получения файла размером 9.1 КБ в зависимости от количества запросов.

\section{Результаты исследования}

Результаты нагрузочного тестирования разработанного сервера и nginx представлены в таблице~\ref{tab:1}.

\begin{table}[H]
	\centering
	\caption{RPS при обработке 10000 запросов на файл размером 2 Кб}
	\label{tab:1}
	\begin{tabular}{|c|c|c|}
		\hline
		Число запросов & Разработанный сервер & nginx \\ \hline
		10 & 414.97 & 406.74 \\ \hline
		50 & 1878.45 & 1823.23 \\ \hline
		100 & 1798.11 & 1155.56 \\ \hline
		500 & 1798.11 & 1155.56 \\ \hline
		1000 & 1798.11 & 1155.56 \\ \hline
		5000 & 1798.11 & 1155.56 \\ \hline
		10000 & 1798.11 & 1155.56 \\ \hline
	\end{tabular}
\end{table}

\section*{Выводы}