\chapter{Исследовательский раздел}

\section{Технические характеристики и описание исследования}

Технические характеристики устройства, на котором выполнялось тестирование:
\begin{enumerate}
	\item операционная система~---~macOS 15.1.1 (24B91); 
	\item объем оперативной памяти~---~18 Гбайт;
	\item процессор~---~Apple M3 Pro, 11 ядер.
\end{enumerate}

Во время тестирования ноутбук был подключен к сети электропитания и нагружен только встроенными приложениями и системой тестирования.

В качестве стороннего веб-сервера для сравнения производительности был выбран \texttt{nginx}, так как является самым популярным и востребованным.
В качестве инструмента для генерации нагрузки и замеров производительности будет использоваться утилита \texttt{ab} (Apache Benchmark).
Целью исследования является сравнение реализованного веб-сервера с известными аналогами по среднему времени ответа на получение файла размером 9.1 КБ в зависимости от количества запросов~\cite{nginx, ab}.

\clearpage
\section{Результаты исследования}

Результаты сравнения веб-серверов представлены в таблице~\ref{table1}.
\begin{table}[!ht]
	\centering
	\caption{Среднее время ответа на запрос для получения файла}
	\label{table1}
	\begin{tabularx}{\textwidth}{|X|X|X|}
		\hline
		Количество запросов & Среднее время ответа (разработанный веб-сервер), мс & Среднее время ответа (nginx), мс \\ \hline
		100 & 0.209 & 0.739 \\ \hline
		1000 & 0.091 & 0.460 \\ \hline
		5000 & 0.079 & 0.440 \\ \hline
		10000 & 0.077 & 0.444 \\ \hline
		50000 & 0.077 & 0.444 \\ \hline
		100000 & 0.077 & 0.444 \\ \hline
	\end{tabularx}
\end{table}

\section*{Вывод}

В исследовательском разделе было проведено сравнение реализованного статического веб-сервера и веб-сервера \texttt{nginx} по времени получения файла размером 9.1 КБ.
Исходя из полученных в таблице~\ref{table1} результатов, был сделан вывод, что время ответа для небольшого количества запросов (до 1000) примерно в 2 раза больше, чем для большого количества запросов (больше 1000).
При этом разница во времени ответа между разным количеством запросов уменьшается с повышением самого числа запросов для обоих серверов.
Реализованный веб-сервер в среднем превосходит \texttt{nginx} по скорости ответа примерно в 3.5 раза до 1000 запросов и примерно в 6 раз для количества запросов, превышающего 1000.