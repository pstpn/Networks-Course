\chapter{Аналитический раздел} 

\section{Архитектура thread-pool}

Архитектура пул потоков (thread-pool)~---~модель управления потоками выполнения (threads), которая предусматривает предварительное создание фиксированного или динамически изменяемого пула потоков, используемых для обработки задач из общей очереди.
Такая архитектура обеспечивает эффективное распределение вычислительных ресурсов и уменьшение накладных расходов, связанных с созданием и уничтожением потоков.
Она является одной из самых распространенных архитектур многопоточности Object Request Broker Architecture (CORBA), используемых в реализациях Object Request Broker (ORB), и была принята веб-серверами, такими как Microsoft Internet Information Server (IIS)~\cite{tp}.

Основными компонентами архитектуры thread-pool являются:
\begin{enumerate}
	\item пул потоков (thread-pool);
	\item очередь задач (task queue);
	\item менеджер пула (pool manager);
	\item задачи (tasks).
\end{enumerate}

\clearpage
Архитектура пула потоков представлена на рисунке~\ref{img:threadpool}.
\includeimage
	{threadpool}
	{f}
	{H}
	{1\textwidth}
	{Архитектура пула потоков}

\section{Системный вызов \texttt{pselect}}

В Unix процесс выполняет ввод-вывод по одному файловому дескриптору за раз, поэтому происходит блокировка и снижение производительности программы.
Чтобы избежать этой проблемы, необходимо использовать системный вызов \texttt{pselect()}, который позволяет программе отслеживать несколько файловых дескрипторов.
Программа ожидает, пока один или несколько файловых дескрипторов не станут готовы к определённому классу операций ввода-вывода, не блокируя их и обеспечивая многопоточный синхронный ввод-вывод~\cite{pselect}.

\clearpage
Модель неблокирующего синхронного ввода-вывода, в которой применяется \texttt{pselect()}, представлена на рисунке~\ref{img:iomodel}.
\includeimage
	{iomodel}
	{f}
	{H}
	{1\textwidth}
	{Модель неблокирующего синхронного ввода-вывода}