\chapter*{ВВЕДЕНИЕ}
\addcontentsline{toc}{chapter}{ВВЕДЕНИЕ}

Статический веб-сервер~---~сервер, который обслуживает статические файлы по запросу клиента.
Cтатические файлы~---~файлы, содержимое которых не изменяется динамически на стороне сервера.
Статический сервер предназначен для чтения файлов из диска и отправления их клиенту, как правило, по протоколу HTTP.

Целью курсовой работы является разработка сервера для отдачи статического содержимого с диска.
Архитектура сервера должна быть основана на пуле потоков (thread-pool) совместно с\texttt{pselect()}.

Для достижения поставленной цели необходимо решить следующие задачи:
\begin{enumerate}
	\item описать предметную область;
	\item спроектировать схему алгоритма работы сервера;
	\item выбрать средства реализации сервера;
	\item провести сравнение реализованного сервера с известными аналогами.
\end{enumerate} 
